\documentclass[titlepage,11pt, oneside]{article}   	% use "amsart" instead of "article" for AMSLaTeX format
\usepackage{geometry}                		% See geometry.pdf to learn the layout options. There are lots.
\usepackage[utf8]{inputenc}
\geometry{letterpaper}                   		% ... or a4paper or a5paper or ... 
%\geometry{landscape}                		% Activate for rotated page geometry
%\usepackage[parfill]{parskip}    		% Activate to begin paragraphs with an empty line rather than an indent
\usepackage{graphicx}				% Use pdf, png, jpg, or eps§ with pdflatex; use eps in DVI mode
								% TeX will automatically convert eps --> pdf in pdflatex		
\usepackage{amssymb}
\usepackage{authblk}
\usepackage{indentfirst}
\usepackage{gensymb}
\usepackage{hyperref}
%SetFonts


\title{\textbf{Complete chloroplast genome sequence of a white spruce (\textit{Picea glauca}) genotype from eastern Canada}}
\author{TBD}
\date{February 11, 2019}					% Activate to display a given date or no date

\begin{document}
\maketitle
\begin{abstract}
\textit{Picea glauca}, is a white spruce tree native to eastern Canada. Here we present the complete chloroplast genome sequence of a \textit{Picea glauca}, isolate WS77111. This sequence will contribute to current efforts made in the study of evolutionary phylogeny of the \textit{Picea} trees, subsequently benefiting Canada’s forestry industry.

\end{abstract}

\section*{Genome Announcement}
As a part of the \href{http://spruce-up.ca/en/}{SpruceUp} (1) and \href{https://www.smartforests.ca}{SMarTForests} (2) projects, we sequenced, assembled and annotated the chloroplast genome of \textit{Picea glauca}, isolate WS77111. This work contributes to improving the genomic selection in spruce breeding programs, a vital part of Canada’s forestry industry.
\newline
\par
The \textit{Picea glauca} isolate WS77111 tissue sample of the needles was collected in southern Ontario (44\degree19'48"N, 78\degree9'0"W; elevation: 250m), from Universit\'{e} Laval. Subsequently, the sample was sequenced at the \href{http://www.bcgsc.ca}{British Columbia Genome Sciences Centre} (BCGSC).
\newline
\par
To sequence the sample, genomic DNA libraries were constructed according to BCGSC plate-based and paired-end library protocols on a Microlab NIMBUS liquid handling robot (Hamilton, USA). Briefly, 1$\mu$g of high molecular weight genomic DNA was sonicated (Covaris LE220) in 62.5$\mu$L volume to 400-bp. Sonicated DNA was purified with PCRClean DX magnetic beads (Aline Biosciences). The DNA fragments were end-repaired, phosphorylated and bead-purified in preparation for A-tailing using a custom NEB Paired-End Sample Prep Premix Kit (New England Biolabs). Illumina sequencing adapters were ligated overnight at 16\degree C while adapter ligated products were bead purified and enriched with 6 cycles of PCR using primers containing a hexamer index that enables library pooling. Pooled libraries were sequenced with paired-end 250-bp reads on an Illumina HiSeq2500 instrument in rapid mode.
\newline
\par
For assembly, we subsampled the reads into a few million read pairs. ABySS v2.1.0 (3) assembled each subset. Then, BWA v0.7.17 (4) filtered for contigs greater than 500-bp aligning with the reference chloroplast, \textit{Picea glauca} isolate PG29. QUAST v5.0.0 (5) revealed that the 1.5M subset contained zero misassemblies (\textit{n}=14; \textit{N\textsubscript{50}}=18371), while the 6M subset contained the least contigs (\textit{n}=6; \textit{N\textsubscript{50}}=43759). Therefore, we selected the 1.5M, 6M and 3M subset (middle ground) for a parameter sweep, where the 3M subset was eliminated. After LINKS v1.8.5 (6), both assemblies combined the contigs into one piece. QUAST determined that the 1.5M subset contained 12 gaps, whereas the 6M contained five. We then employed ABySS-sealer to close the gaps, where the 1.5M subset contained the fewest gaps. We maintained consistency by modifying our assembly to match the reference using BLAST v.2.7.1 (7) and polished the final assembly with Pilon v1.22 (8). 
\newline
\par
The WS77111 chloroplast genome is 123,421-bp in length, with 38.74\% GC content. GeSeq (9) annotated 114 genes: 74 protein-coding, 36 tRNA-coding, and four rRNA-coding genes, using other  \textit{Picea} chloroplast genomes as reference. Only \textit{rps12}, \textit{petB}, \textit{petD}, \textit{rpl16}, and \textit{psbZ} required manual annotation. OGDRAW v1.2 (10) drew the genome map.  The addition of this new chloroplast genome to resource pools will enable further analysis of \textit{Picea} phylogeny and evolution.
\newline
\newline
\textbf{Accession number(s).} The complete chloroplast genome sequence of \textit{Picea glauca}, isolate WS77111 can be found in Genbank under MK174379. The tissue samples used can be found with BioSample: \href{https://www.ncbi.nlm.nih.gov/biosample/?term=SAMN02736787}{SAMN02736787} and BioProject: \href{https://www.ncbi.nlm.nih.gov/bioproject/?term=PRJNA242552}{PRJNA242552}. The annotation NCBI references are as follows: \textit{Picea abies} (\href{https://www.ncbi.nlm.nih.gov/nuccore/NC_021456}{NC\_021456}), \textit{Picea asperata} (\href{https://www.ncbi.nlm.nih.gov/nuccore/NC_032367}{NC\_032367}), \textit{Picea glauca} isolate PG29 (\href{https://www.ncbi.nlm.nih.gov/nuccore/NC_028594}{NC\_028594}), \textit{Picea morrisonicola} (\href{https://www.ncbi.nlm.nih.gov/nuccore/NC_016069}{NC\_016069}), and \textit{Picea sitchensis }(\href{https://www.ncbi.nlm.nih.gov/nuccore/NC_011152}{NC\_011152}).

\section*{Acknowledgements}
This work was conducted as a part of the SpruceUp and SMarTForests projects.

\section*{References}

\begin{enumerate}
\setcounter{enumi}{2}
\item Jackman SD, Vandervalk BP, Mohamadi H, Chu J, Yeo S, Hammond SA, Jahesh G, Khan H, Coombe L, Warren RL, Birol I. 2017. ABySS 2.0: resource-efficient assembly of large 
genomes using a Bloom filter. Genome research 27:768-777.
\item Li H, Durbin R. 2009. Fast and accurate short read alignment with Burrows-Wheeler transform. Bioinformatics (Oxford, England) 25:1754-1760.
\item Gurevich A, Saveliev V, Vyahhi N, Tesler G. 2013. QUAST: quality assessment tool for genome assemblies. Bioinformatics (Oxford, England) 29:1072-1075.
\item Warren RL, Yang C, Vandervalk BP, Behsaz B, Lagman A, Jones SJM, Birol I. 2015. LINKS: Scalable, alignment-free scaffolding of draft genomes with long reads. GigaScience 4:35-35.
\item Altschul SF, Gish W, Miller W, Myers EW, Lipman DJ. 1990. Basic local alignment search tool. Journal of Molecular Biology 215:403-410.
\item Walker BJ, Abeel T, Shea T, Priest M, Abouelliel A, Sakthikumar S, Cuomo CA, Zeng Q, Wortman J, Young SK, Earl AM. 2014. Pilon: an integrated tool for comprehensive microbial variant detection and genome assembly improvement. PloS one 9:e112963-e112963.
\item Tillich M, Lehwark P, Pellizzer T, Ulbricht-Jones ES, Fischer A, Bock R, Greiner S. 2017. GeSeq - versatile and accurate annotation of organelle genomes. Nucleic acids research 45:W6-W11.
\item Lohse M, Drechsel O, Kahlau S, Bock R. 2013. OrganellarGenomeDRAW--a suite of tools for generating physical maps of plastid and mitochondrial genomes and visualizing expression data sets. Nucleic acids research 41:W575-W581.
\end{enumerate}

\end{document}  